\chapter{Cooperative Fuzzing Framework}
In this chapter we describe the design of our \acf{CFF} and details about its
implementation. We are going to show how instances of different fuzzers can
communicate with each other in order to share knowledge and cooperate to improve
the overall fuzzing efficacy (in terms of coverage and number of unique bugs
found).

\section{System Design}
% describe design goals and how it works
The \ac{CFF} is designed to run fuzzers of different
natures in parallel, harvest knowledge from each instance and intelligently
broadcast it to selected other instances. This framework allows construction of
algorithms that orchestrate the information flow among the fuzzing instances. We
design two primitives that allow a fuzzer to share knowledge with other fuzzers
and devise a distributed architecture that employs these primitives to manage
the information flow among fuzzer instances in a smart manner. The architecture
is modular so that inner components can be replaced as needed, moreover it can
be deployed across multiple machines.

\subsection{Common Fuzzer Interface}
\label{sec:system-design-api}
% need for API / common layer for fuzzers
We require fuzzers to implement two primitives that allow our framework to
interface with them. Having a common interface enables the framework to operate
with different fuzzers, as long as they adhere to the \ac{API}.

All fuzzers work with test-cases and \acp{CGF} keep a queue of test-cases adding
elements to it if they (loosely speaking) increase coverage. Whenever a fuzzer
finds a test-case that expresses some interesting property, stores it into the
internal queue. This property is deemed interesting in a subjective manner,
based on how the fuzzer operates and the knowledge it has already acquired about
the \ac{SUT}. It follows that we can track the progress of a fuzzer by
intercepting and analysing interesting test-cases; in other words we can
\textbf{extract} knowledge from the fuzzer. The first primitive that we require
a fuzzer to expose to the \ac{API} enables exactly this: we require a fuzzer to
store interesting test-cases to the file system. This choice was made since most
\acp{CGF} already fulfill this requirement; this is because the resulting corpus
can be reused for future fuzzing campaigns against the same software or another
that accepts the same input format without starting from scratch.

The second primitive we require a fuzzer to implement, allows our cooperative
framework to \textbf{inject} test-cases into it. This mechanic is inspired by
that of pollination, where pollen carrying the genetic material of one plant is
transferred to another plant to be later fertilized. The injection primitive in
our framework allows one test-case, carrying some knowledge about the fuzzer
instance it comes from, to be merged with the knowledge of the receiving fuzzer
instance. As with the extraction primitive, injection operates through the file
system: the fuzzer needs to periodically check a predetermined folder and import
new test-cases into memory.

Different fuzzers may work at different paces: traditional \acp{CGF} \emph{tick}
after each new test-case is generated; fuzzers that follow the evolutionary
approach advance a time-step when a new generation gets synthesized (\eg~after
several test-cases are generated). In between successive time-steps, a fuzzer is
not able to accept and process external test-cases. Thus there is a need for a
fuzzer to communicate through the cooperative framework when it is able to
accept new input (\eg~when the inject primitive can be called on this fuzzer
instance). We assume that a fuzzer is always able to accept new test-cases,
otherwise it needs to implement a third interface endpoint: when the fuzzer can
accept new input, it will broadcast this information to all other connected
fuzzers. This information allows for a form of \textbf{congestion control}
within our cooperative framework and is entirely delegated to the fuzzer in
order to abstract over its implementation details.

With these two primitives in place (not counting the latest introduced, which is
optional), our cooperative framework is capable of interacting with the running
fuzzer instances by directly accessing an uniform communication layer that uses
test-cases as an atomic piece of information. Table~\ref{tab:system-primitives}
summarizes the \ac{CFF} interface with fuzzer instances.

\begin{table}[h]
    \centering
    \begin{tabularx}{\textwidth}{X >{\raggedright}p{.3\textwidth} X p{.15\textwidth}}
        \textbf{Primitive} & \textbf{Description} & \textbf{Initiated by} & \textbf{Required} \\
        \bottomrule
        Extract & Extracts interesting test-cases from fuzzer & Fuzzer & Yes \\
        Inject & Injects test-cases into fuzzer & Framework & Yes \\
        Congestion Control & Signals to the framework when the fuzzer is able to
            consume injected test-cases & Fuzzer & No
    \end{tabularx}
\caption{\acl{CFF} interface with fuzzers}
\label{tab:system-primitives}
\end{table}

\subsection{Central Decisional Unit}
\label{sec:system-design-cdu}
% need for a central decisional unit (mention strategies)
With a common interface with fuzzers in place, the next component we need to
devise is a decisional unit responsible of controlling the information flow
among fuzzer instances. This component acts as an intermediary or a filter for
the interface between two fuzzers, applying a pre-determined strategy that
decides which test-cases extracted from a fuzzer should be injected into which
fuzzer (if any). As some fuzzers need some form of congestion control, the
decisional unit is also responsible of deferring the call to the injection
primitive. As an example, when an evolutionary \ac{CGF} is in the set of fuzzer
instances, one strategy might choose to keep track of the best (according to
some metric) test-case extracted from the other running fuzzers, to inject it
only when the evolutionary fuzzer is able to consume injected test-cases;
another strategy might choose to do the same only for a subset of running
fuzzers and so on. We note that for strategies such as those briefly described
above, we need the decisional unit to be centralized: communication does not
happen between fuzzer instances directly, but everything is coordinated by a
\textbf{central decisional unit} that models communication in accordance to an
user-defined strategy.

\begin{figure}
    \centering
    \subfloat[]
        [Physical view of the framework communication model. Running fuzzers
        communicate only with the central decisional unit which selectively
        broadcasts messages to other fuzzers]{%
            \label{fig:system-comm-phy}
            \includegraphics[width=.6\textwidth]{figures/dia/system_design_physical}}%
    \qquad
    \subfloat[]
        [Logical view of the framework communication model. Information can flow
        across all running fuzzers]{%
            \label{fig:system-comm-log}
            \includegraphics[width=.6\textwidth]{figures/dia/system_design_logical}}
\caption{\acl{CFF} communication model}
\label{fig:system-comm}
\end{figure}

Figure~\ref{fig:system-comm} gives a graphical representation of the
communication model in the \ac{CFF}. More specifically,
Figure~\ref{fig:system-comm-phy} depicts the \emph{physical} links between
components of the framework; Figure~\ref{fig:system-comm-log} instead shows how,
on the \emph{logical} level, each running fuzzer can directly exchange
test-cases with other running fuzzers. In this configuration each running fuzzer
communicates directly only with the central decisional unit by means of the
common interface described in Section~\ref{sec:system-design-api}, without
having any knowledge about other running fuzzers.

\subsection{Cooperative Fuzzing Strategies}
\label{sec:system-design-strats}
% describe strategies for cooperation and metrics
With the architectural design in place, we can proceed to describe a more
detailed view on the \ac{CFF} operation.

\begin{algorithm}
    \DontPrintSemicolon
    \SetKwFunction{Score}{Score}
    \SetKwFunction{Winning}{Winning}
    \SetKwFunction{Inject}{Inject}
    \SetKwFunction{WinningC}{WinningCongestion}
    \SetKwInOut{Input}{Input}\SetKwInOut{Output}{Output}
    \Input{Set of running fuzzers $F$. Set of fuzzers that need congestion control $F_c$}
    \BlankLine
    \ForEach{$f \in F_c$}{%
        $W_f \leftarrow \emptyset$\;
    }
    \While{all fuzzers are running}{%
        \If{new test-case $t$ from a fuzzer $f_t$}{%
            $S \leftarrow \emptyset$\;
            \ForEach{$f \in F \setminus \{f_t\}$}{%
                $s \leftarrow \Score{f, t}$\;
                \uIf{$f \in F_c$}{%
                    $W_f \leftarrow W_f \cup \{(t, s)\}$\;
                }
                \Else{%
                    $S \leftarrow S \cup \{(f, s)\}$\;
                }
            }
            \ForEach{$f \in \Winning{S}$}{%
                \Inject{f, t}\;
            }
        }
        \ForEach{$f \in F_c$}{%
            \If{$f$ is ready to receive inputs}{%
                \ForEach{$t \in \WinningC{$W_f$}$}{%
                    \Inject{f, t}\;
                }
                $W_f \leftarrow \emptyset$\;
            }
        }
    }
\caption{Generic strategy for the \acl{CFF}}
\label{algo:system-strategy}
\end{algorithm}

Algorithm~\ref{algo:system-strategy} gives the generic template of a cooperative
fuzzing strategy. There are three user-defined functions that describe which
test-cases should be forwarded to which fuzzer (in other words they define the
strategy). When a fuzzer stores a new interesting test-case, the extract
primitive gets executed and the cooperative framework will pick it up. Then, for
each other running fuzzer, the \texttt{Score} function will assign a numeric
value representative of some measurement on the test-case in relation to the
fuzzer's state. These numeric values are then used in two \emph{winner
selection} procedures: \texttt{Winning} selects a set of fuzzers that are not
under congestion control into which the newly extracted test-case is going to be
injected; \texttt{WinningCongestion} instead, for each fuzzer under congestion
control which is ready to receive new inputs, selects a set of test-cases within
the window set $W_f$ of inputs collected since the previous round.

% describe requirements for a valid metric
The \texttt{Score} function of Algorithm~\ref{algo:system-strategy} is a core
component around which the design process of a cooperative strategy rotates.
When devising a new strategy, one would start by defining a scoring function as
the winner selection procedures will use its values to make the final decision.
The scoring function is fundamental from another point of view: it has been
designed to be analogous to the fitness function found in \acp{EA}, where in our
case \emph{fitness} represents how much would a fuzzer benefit from receiving a
given test-case.

% examples of metric / strategy
As for \acp{CGF}, the \ac{CFF} uses code coverage as a basis for the scoring
function. A strategy might choose to keep track of all basic blocks discovered
by each running fuzzer (this is possible because \emph{interesting} test-cases
often execute previously unseen basic blocks); the scoring function can than
return a numeric value representing a specific property of the newly extracted
test-case in relation to the fuzzer's current knowledge (the set of discovered
basic blocks). Behind the intuition that seeding a fuzzer with a test-case that
exhibits new basic blocks might help the fuzzer discovering even more new basic
blocks, we might construct a strategy to broadcast test-cases to those fuzzers
for which there is at least one undiscovered basic block. In this context, we
can set the scoring function to return the number of basic blocks new to the
fuzzer and select as winners fuzzers with a positive score. % TODO Jaccard

% describe possible winning conditions
The \texttt{Winning} function of Algorithm~\ref{algo:system-strategy} is
responsible for working out a set of running fuzzers, given the results of the
scoring function. Fuzzers from this set, which can be empty, are to be injected
with the test-case. This function resembles the selection stage of \acp{EA}:
selects fuzzers (in contrast to selecting individuals of a population) which are
going to receive a new test-case, so that the fuzzer itself can apply its own
mutation algorithms to try to generate new interesting test-cases. The
\texttt{WinningCongestion} function on the other end, instead of working with a
set of scores for different fuzzers over a single test-case, selects a set of
scored test-cases to be injected into a single fuzzer.

\section{System Implementation}
% describe uberfuzz implementation
This section describes our implementation of the \ac{CFF}, which is composed of
two executables: a \emph{driver} and a \emph{master}. A driver is responsible to
run, monitor and interact with a fuzzer instance (\ie~via the \ac{API} defined
in Section~\ref{sec:system-design-api}) while the master implements the central
decisional unit as defined in Section~\ref{sec:system-design-cdu}; together,
drivers and master, implement the \ac{CFF} strategy described in
Section~\ref{sec:system-design-strats}. The full implementation, which is
available on Github\footnote{\url{https://github.com/acidghost/uberfuzz2}},
consists of a total of $2,832$ lines of Rust, C and Bash source code.

\subsection{Communication Channels}
Driver and master are implemented as two separate executables to fulfill the
requirement of being able to deploy the different components across different
machines. To support this, drivers communicate with the master using three
separate distributed message queues as depicted in Figure~\ref{fig:system-impl};
these communication channels are implemented over
ZeroMQ\footnote{\url{http://zeromq.org/}}, a popular asynchronous messaging
library with binding for a large variety of languages, including Rust and C. All
messages over these channels are serialized as space-separated values. The
interesting channel is a queue shared among all drivers on which interesting
inputs captured by drivers are pushed to be later pulled by the master. Messages
on this channel have the following fields:

\begin{description}
    \item[\texttt{fuzzer\_id}] identifies the fuzzer instance that generated the
        input;
    \item[\texttt{input\_path}] the path to the input deemed interesting by the
        fuzzer;
    \item[\texttt{coverage\_path}] path to a binary file containing the coverage
        information generated by executing the \ac{SUT} with the input at
        \texttt{input\_path}.
\end{description}

The inject channel is, again, shared among all drivers and is set up so that
master publishes messages to which drivers can subscribe. Messages on this
channel represent inputs that specific fuzzers need to inject into their queue
of working inputs. Each message is made up of the following fields:

\begin{description}
    \item[\texttt{fuzzer\_ids}] a list of fuzzer identifiers, separated by '\_',
        that need to receive and inject the input represented by this message;
    \item[\texttt{input\_path}] path to the input to be injected;
    \item[\texttt{coverage\_path}] path to coverage data exercised by the given
        input.
\end{description}

\begin{figure}[t]
    \centering
    \includegraphics[width=.9\textwidth]{figures/dia/system_impl_channels}
    \caption{Communication channels between fuzzer, driver and master.}
    \label{fig:system-impl}
\end{figure}

Finally, the metrics channel uses the request-reply pattern: each driver binds
to a unique port (\ie~the channel is not shared among drivers) in reply mode and
the master connects to it in request mode. Whenever the master needs a metric
evaluation from a specific fuzzer, it sends a metric request to the appropriate
driver and synchronously waits for a reply. Both request and reply messages are
comprised of a single field: requests contain a \texttt{coverage\_path}, as
described for other message types; replies contain instead a floating point
number, representing the metric evaluation result.

Communication between a fuzzer and its driver is achieved through the file
system. This is because most fuzzers use the file system to store their data.
More specifically ``interesting'' inputs, as well as inputs to be fuzzed, are
written to a folder (which might be the same for both). Follows that to capture
new interesting inputs, a driver needs to track new files added to the
appropriate folder. On the other hand, to inject new inputs into the fuzzer's
queue, a driver simply needs to add a file with the input to the right folder.
The Linux \texttt{inotify} subsystem is used to avoid doing inefficient long
polling to capture fuzzer events (\eg~when a new interesting input is added).
For fuzzers that need congestion control, we require them to write (\ie~fuzzers
listen for \texttt{inotify} \texttt{IN\_MODIFY} events) to a predetermined file
whenever there is availability to receive new inputs.

\subsection{Driver Implementation}
\label{sec:driver-impl}

\subsection{Master Implementation}
\label{sec:master-impl}

