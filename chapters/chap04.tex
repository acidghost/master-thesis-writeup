\chapter{Evaluation}
\label{chap:evaluation}

% compare:
% - mono (no best fuzzer for sample programs)
% - multi vs single (more communication is better if multi > single)
% - coop vs union (test coop efficacy)

This chapter presents an evaluation of four different fuzzers and our
implementation of the \acl{CFF} as described in Section~\ref{sec:system-impl};
the implementation uses the same four fuzzers in order to draw meaningful
comparison results.

\paragraph{Choice of Fuzzers}

For the choice of fuzzers we decided to use \aflfast, \fairfuzz, \honggfuzz~and
\vuzzer. The first two provide two different implementations of the
state-of-the-art \ac{CGF}, \afl; we argue that two different implementation of
the same core algorithm may yield different results. \honggfuzz~is chosen as it
provides different means to extract feedback from the \sut, possibly resulting
in a different feedback signal (given the same input) compared to other methods
(\eg~\afl~uses QEMU). Lastly, \vuzzer~provides with an evolutionary fuzzer which
uses static and lightweight program analysis; because of this, \ac{CFF} uses the
congestion control mechanism to interact with \vuzzer.

\paragraph{Experimental Infrastructure}

Experiments were run on a 64-bit machine with $8$ cores (running at $3.4$GHz),
with $16$GB of RAM, running Ubuntu 16.04. Because of a limitation of \vuzzer\@
implementation, we had to run it inside a virtual machine hosting Ubuntu 14.04.
To allow communication with its driver running on the host machine, we used a
shared folder; this allowed us to reuse the \texttt{inotify} infrastructure as
if the fuzzer was running locally.

\paragraph{Testing Targets}

We chose a set of programs that are widely used both in practice and literature:

\begin{description}
    \item[\djpeg] uses the popular \texttt{libjpeg-turbo} (version 1.5.1) and
        has been used for the evaluation of \fairfuzz~and \vuzzer;
    \item[\objdump] is a component of the suite \texttt{binutils} (version 2.28)
        which has also been tested by \aflfast~and \fairfuzz; we test it with
        the command line option \texttt{-d}, which provides a disassembler;
    \item[\tiffpdf] from the popular \texttt{libtiff} (version 4.0.9), parses a
        TIFF image and converts it to a PDF\@; we use it without command line
        arguments;
    \item[\listswf] from the popular \texttt{libming} (version 0.4.8), parses a
        file in SWF format; we use it without command line arguments.
\end{description}

\section{Single Fuzzer Evaluation}

In this section we present an evaluation of the chosen fuzzers, without any
cooperation. We ran each fuzzer independently for $24$ hours on each of the
targets, except for \listswf~which ran for $6$ hours. Table~\ref{tab:eval-mono}
reports the arithmetic mean and the $95\%$ confidence intervals for the number
of unique basic block transitions as computed by Intel \ac{BTS} over five
rounds.

\begin{table}[h]
    \centering%
    \small%
    \begin{adjustbox}{tabular=l*{4}c,center}
        \textbf{\sut} & \textbf{\aflfast} & \textbf{\fairfuzz} &
            \textbf{\honggfuzz} & \textbf{\vuzzer} \\
        \bottomrule%
        \djpeg& $3739.4 \pm 113.831$ & $4043.2 \pm 103.838$ &
            \hicell$4112.8 \pm 39.5483$ & $2801 \pm 53.5514$ \\
        \objdump& $4762.6 \pm 23.3693$ & \hicell$5067 \pm 62.6832$ &
            $4132.4 \pm 104.8$ & $3162.2 \pm 138.462$ \\
        \tiffpdf& \hicell$8971.2 \pm 152.865$ & $8813.8 \pm 146.756$ &
            $5260.2 \pm 148.591$ & $3616 \pm 34.6427$ \\
        \listswf& $6831.6 \pm 2615.24$ & \hicell$8586.8 \pm 87.7467$ &
            $6345.6 \pm 2358.52$ & $5048.2 \pm 90.3928$
    \end{adjustbox}
    \caption{Mean coverage with $95\%$ confidence intervals for single fuzzers.
    Highlighted is the best for the given program.}
    \label{tab:eval-mono}
\end{table}

\begin{figure}[h]
    \centering%
    \begin{adjustbox}{center}
        \subfloat[\djpeg]{%
            \includegraphics[width=.65\textwidth]{figures/mono-djpeg}
            \label{fig:eval-mono-djpeg}
        }
        \subfloat[\objdump]{%
            \includegraphics[width=.65\textwidth]{figures/mono-objdump}
            \label{fig:eval-mono-objdump}
        }
    \end{adjustbox}
    \begin{adjustbox}{center}
        \subfloat[\tiffpdf]{%
            \includegraphics[width=.65\textwidth]{figures/mono-tiff2pdf}
            \label{fig:eval-mono-tiff2pdf}
        }
        \subfloat[\listswf]{%
            \includegraphics[width=.65\textwidth]{figures/mono-ming}
            \label{fig:eval-mono-ming}
        }
    \end{adjustbox}
    \caption{Mean coverage over time for single fuzzers.}
    \label{fig:eval-mono}
\end{figure}

\section{Cooperative Fuzzing Evaluation}

\begin{table}[h]
    \centering%
    \begin{tabular}{l c c c}
        \textbf{\sut} & \textbf{multi} & \textbf{single} & \textbf{union} \\
        \bottomrule%
        \djpeg& $4056.4 \pm 76.9499$ & \hicell$4078.4 \pm 85.6738$ & $4028.6 \pm 47.7396$ \\
        \objdump& $5414.6 \pm 224.121$ & \hicell$5529.6 \pm 338.651$ & $5035.6 \pm 54.5944$ \\
        \tiffpdf& \hicell$8765.6 \pm 183.682$ & $8577.6 \pm 99.2457$ & $8623.2 \pm 183.399$ \\
        \listswf& \hicell$9008.4 \pm 122.81$ & $8801.4 \pm 96.4045$ & $8916.6 \pm 83.8381$
    \end{tabular}
    \caption{Mean coverage with $95\%$ confidence intervals for winning
    strategies that select single or multiple winners and without cooperation.}
    \label{tab:eval-single-multi}
\end{table}

\section{Crash Analysis}

