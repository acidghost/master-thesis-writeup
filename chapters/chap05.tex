\chapter{Discussion}
\label{chap:discussion}

This chapter contains a discussion of the evaluation results presented in
\autoref{chap:evaluation}. With regards to our research question of whether for
fuzzing the No Free Lunch Theorem holds, \autoref{sec:eval-mono} showed that
some fuzzers perform decisively better than the other on some programs while
perform poorly for other. This confirms our initial hypothesis that there are no
free lunches for fuzzing.

In \autoref{sec:eval-coop} we tried to establish whether introducing cooperation
into a group of fuzzers running in parallel proves beneficial. We tackled the
question from two points of view: coverage and crashes found. When analyzing
coverage results we were unable to declare a winner with credibly high
certainty. Bayesian estimation could not provide strong evidence supporting the
hypothesis that cooperation is beneficial; the results are nonetheless promising
and one missing key ingredient to obtain a more decisive result, in our opinion,
is more data. With more data, Bayesian estimation would return a picture of the
difference of means that reflects more the true distribution, giving more space
to the possibility to draw a confidant conclusion. Moreover we are unable to
find a decisive winner among the two cooperative strategies although we note
that the multiple winners strategy performs better than the union of fuzzers
even when the single winner strategy performs the best.

The analysis of crashes presented in \autoref{sec:eval-crashes} reveals an image
more in favour of cooperation. The cooperative strategy that uncovered the most
basic block transitions uncovers also the most unique crashes. Furthermore the
multiple winners cooperative strategy uncovers more distinct unique crashes than
the union of fuzzers with a factor of $1.3$. Also, it finds all \acp{CVE} that
the union of fuzzers finds, plus two more that are not found by the union.

