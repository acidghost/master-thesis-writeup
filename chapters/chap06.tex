\chapter{Future Work}
\label{chap:future}

In this chapter we discuss possible improvements to the \ac{CFF} and its
evaluation. To begin with, one possible line of research for future improvements
may be to devise a more complex strategy that can exploit the \ac{BTS} trace to
the fullest. For example, one strategy could reason about the path exercised by
a test case and compare it to the tree of already-discovered paths; a similarity
metric can be returned to the master which will decide to which fuzzer to inject
the test case. The similarity metric should be crafted so that higher values are
associated to the exercised path which is closer to the discovered tree of the
fuzzer. The intuition behind this is that the injection of such test cases may
allow a fuzzer to better explore its vicinity. On the other hand one could
devise a metric based on dissimilarity with the aim of injecting test cases that
differ the most from the discovered tree to broaden the search and possibly
escape local maxima.

Another possible extension to the present work is to broaden the spectrum of
fuzzers used to evaluate the \ac{CFF}. For example, the inclusion of a black box
fuzzer wouldn't require any modification to the current implementation of the
\ac{CFF} to integrate it into the roster of fuzzers. A fuzzer of this kind would
provide with much more throughput in terms of number of \sut\ executions given
that no instrumentation is used; because of this though, a black box fuzzer is
missing the notion of interesting input and its interaction with the \ac{CFF}
(only in the direction from the fuzzer to the \texttt{driver}) would need to be
designed. To circumvent this, a black box mutational fuzzer may also be used
only to inject new test cases (produced by other, more application-aware
fuzzers) and not to extract them. The inclusion of a purely white box solution
such as a symbolic execution engine could also yield interesting results. In
this case, the \ac{CFF} might integrate with it using the congestion control
mechanism as has been done for \vuzzer\ (\ie~by injecting the test case with the
highest score over a time window).

Finally, as already noted in \autoref{chap:discussion}, the evaluation of the
\ac{CFF} would greatly benefit from having more than five runs for each
experiment. This may allow for the Bayesian estimation to provide results on the
difference of means that express less noise (\ie~more decisive results). On the
same note, increasing the time length of the runs (\eg~from 6 hours to 24) would
allow for the fuzzers to explore the \sut\ more and possibly reach and even more
definite plateau in terms of coverage.

