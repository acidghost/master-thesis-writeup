\chapter{Conclusion}
\label{chap:conclusion}

In this work we investigated the efficacy of running different fuzzers on
different programs. Our hypothesis is that, due to the No Free Lunch theorem for
optimization, no best fuzzer exists when their performance is averaged across
all possible programs. To contrast this, we devise a \acf{CFF} to allow
communication and exchange of information between heterogeneous fuzzer instances
running in parallel. We design a set of \acp{API} that fuzzers must comply to in
order for them to interface with our framework. These \acp{API}, which are
usually already implemented by most fuzzers, allow for \emph{extraction} and
\emph{injection} of test cases from and into the fuzzer. A third, optional,
interfacing method allows for a form of \emph{congestion control} for slower
fuzzers (\eg~those using heavier-weight analysis).

For every newly extracted test case, the framework evaluates it in terms of code
coverage with regards to the already discovered execution tree for each of the
running fuzzers (besides the one that found the test case); this evaluation
consists in running the \sut\ with Intel \ac{BTS} enabled in order to obtain an
hardware-generated execution trace in the form of basic block transitions and
then synthesizing a numeric value to represent this evaluation. A \emph{central
decisional unit} is responsible to collect these evaluations and reason about
their returned values to come up with a set of fuzzers that are going to receive
the given test case. The central decisional unit and the evaluation metric
realize a specific \emph{cooperative fuzzing strategy}.

We have implemented a prototype of the \ac{CFF} that uses the number of newly
discovered basic block transitions in a test case as evaluation metric and
supports a configurable winning strategy (\ie~the component responsible to
select the set of fuzzers that are going to receive the new test case). Four
general winning strategies can be configured: highest or lowest evaluation
metric and higher or lower than a predetermined threshold.

In the evaluation, we presented the results of running four fuzzers without
cooperation with the aim of proving our hypothesis of no best fuzzer. We ran
five experiments of 24 hours long on each \sut\ --- for four \acp{SUT} --- and
compared the average obtained coverage over time. The results of Bayesian
estimation of the difference of means provided support to the confirmation of
our hypothesis as, for all considered \acp{SUT}, the best performing fuzzer is
always different.

Then, we evaluated the effect of introducing cooperation by showing results of
running our implementation of the \ac{CFF} for 6 hours on the same \acp{SUT}. We
configured the framework to evaluate two strategies: one selects a single winner
with the highest metric, the other selects all fuzzers that have returned a
metric higher than zero. In the experiments we used \aflfast, \fairfuzz,
\honggfuzz\ and \vuzzer\ as fuzzer components to the \ac{CFF}. We compared the
results from the two cooperative strategies with the union of the results from
running the same four fuzzers without cooperation. Unfortunately, although the
final mean coverage is higher for all programs for one of the cooperative
strategies, we cannot claim a definitive winner as Bayesian estimation of the
difference of means does not support it with strong confidence, revealing
instead a need for more data.

Moreover, we compared the capability of finding unique crashes and known
vulnerabilities of the \ac{CFF} and the union of fuzzers. In particular, the
cooperative strategy that uncovers the most basic block transitions also finds
more unique crashes than the union of fuzzers with a factor of $1.3$.
Furthermore, the \ac{CFF} finds all the \acp{CVE} that the union finds, plus two
more.

Lastly, we provided an overhead evaluation in which we note that the resources
consumed by the components of the \ac{CFF} are not excessive compared to the
normal operation of its fuzzers. This is especially true, given that the present
implementation has not been extensively developed with performance in mind and
should be considered a prototype.

