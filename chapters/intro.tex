\chapter{Introduction}
\label{chap:intro}

% introduce software testing
\ac{ST} is a broad discipline that aims at assessing the quality of a
software product. Engineers are always interested in gauging the robustness,
reliability and security of their products and \ac{ST} provides them with tools
and techniques to achieve that. Its ultimate goal is to discover as much defects,
failures and vulnerabilities as possible, usually within a limited budget (of
time or computing resources).

% black vs. white box approach
Generally \ac{ST} consists in feeding some input to the \ac{SUT} and monitoring
its behaviour to detect any failure. \ac{ST} can be categorized by the knowledge
required by the tester about the \ac{SUT}. In the white-box approach, the inner
workings of the \ac{SUT} are tested, using knowledge about internal data
structures and control-flow, possibly with the help of source code. In the
black-box approach only the front-facing functionalities are tested as no
internal knowledge of the \ac{SUT} is required. It is easy to see how a
white-box approach has the potential to be more efficient as knowledge about the
\ac{SUT} can better direct the input generation process. Unfortunately this
comes at the cost of analyzing the source code, which isn't always available, or
machine code, which is an hard and computationally intensive task. A black-box
approach on the other end can be carried on for example by simply throwing
random inputs at the \ac{SUT}; \emph{fuzzing} is an automated (or
semi-automated) testing technique consisting in feeding some input to the
\ac{SUT}, monitoring for exceptions (\ie~crashes, hangs), create a new input and
reiterate.

% origin of fuzzing & random testing
The term fuzzing dates back to 1988 from a course project for an Advanced
Operating Systems class taught at the University of Wisconsin by Barton Miller
\cite{takanen2008fuzzing}. After noticing how a thunderstorm was able to
scramble his input through a remote terminal and crash the program on the other
end, he decided to propose a project to his students to experiment with random
testing of UNIX utilities. A group of two students succeeded at the given task
and two years later published their findings~\cite{miller1990empirical}. Random
testing was already in use since the 1950s, when programmers would use punched
cards from the trash or decks of random numbers as input to programs
\cite{weinberg2008}.
% maybe include why random testing works & infinite monkey theorem?

% fuzzers today & gray-box fuzzing
Fuzzers today have much evolved and have been used by major companies such as
Microsoft~\cite{godefroid2008grammar} and
Google~\cite{google2016fuzz,google2011fuzz,google2016oss,google2017oss}. Fuzzers
can be categorized along different axes. The way a fuzzer generates its
\emph{fuzz} (test cases) makes it a generation-based or mutation-based fuzzer;
if new fuzz is created from scratch it belongs to the former category, if
instead it is created by changing existing inputs it belongs to the latter. As
for \ac{ST}, fuzzers can have a different level of knowledge of the \ac{SUT} and
then can be classified as white-box or black-box approach.  When it comes to
binaries, a white-box approach usually implies use of heavy program analysis
(\eg~symbolic execution, taint analysis) or manual specification of a grammar to
guide input creation; a black-box approach creates a stream of random inputs to
feed to the \ac{SUT}. A third class has emerged in recent years: gray-box
fuzzing uses lightweight binary instrumentation to monitor the performance of
individual test cases (\eg~through code coverage or other metric) in order to
better guide fuzz creation.

AFL~\cite{afl} is a well known gray-box fuzzer that uses code coverage to know
which inputs are more interesting and focus mutation on those; it is file-based
and uses QEMU\footnote{A processor emulator --- \url{https://www.qemu.org/}} for
binary instrumentation or a compiler extension where source code is available.
Because of its simplicity in design, speed and flexibility, AFL has been
extended in numerous
ways~\cite{bohme2017coverage,bohme2017directed,lemieux2017fairfuzz} or used as a
component of a larger system
\cite{stephens2016driller,nichols2017faster,li2017steelix}. Most coverage-based
gray-box fuzzers work with similar underlying principles that differ in the
implementation. FairFuzz~\cite{lemieux2017fairfuzz} and AFLFast
\cite{bohme2017coverage} use frequencies (one on the branch level, another on
the path level) to guide mutation toward low-frequency entities.
Honggfuzz~\cite{honggfuzz} keeps a queue of inputs that increase code coverage,
randomly picks one and mangles it in a random manner; it also features different
hardware- and software- based feedback sources and because of its efficiency and
versatility has been used as a fuzzing engine of bigger projects
\cite{grieco2016quickfuzz}. VUzzer~\cite{rawat2017vuzzer} differs from the other
fuzzers as its usage of white-box (static analysis and \emph{sparse} taint
analysis) mixed with gray-box techniques constitute an hybrid model.

% TODO: briefly present problems in CGF and argue that there's no best fuzzer

% TODO: brief of this work

