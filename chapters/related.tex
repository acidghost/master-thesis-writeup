\chapter{Related Work}
\label{chap:related}

In this chapter we are going to present the mechanics of some kind of fuzzers
and some specific implementations of \acfp{CGF}. Then we move on presenting
efforts of researchers trying to combine different fuzzing engines or testing
techniques with the aim of improving performance or efficiency.

\section{Black-box Mutational Fuzzing}
\label{sec:bbfuzz}
Black-box \emph{mutational} fuzzing (or \emph{Random Testing}) is a simple
testing technique that uses mutation operators on a sample input to produce a
new input; a corpus of \emph{valid} files (the more the better) is required to
achieve good efficiency by reducing the search space. A simple algorithmic
representation is given in Algorithm~\ref{algo:bbfuzzing}.

\begin{algorithm}
    \DontPrintSemicolon%
    \SetKwFunction{SelectSeed}{SelectSeed}
    \SetKwFunction{MutateSeed}{MutateSeed}
    \SetKwInOut{Input}{Input}\SetKwInOut{Output}{Output}
    \Input{Set of samples $S$}
    \Output{Set of crashing inputs $C$}
    \BlankLine%
    $C \leftarrow \emptyset$\;
    \While{stop condition is not met}{%
        $t \leftarrow \SelectSeed{S}$\;\nllabel{algo:bbfuzzing:ss}
        $t^\prime \leftarrow \MutateSeed{t}$\;\nllabel{algo:bbfuzzing:ms}
        \If{$t^\prime$ crashes program}{%
            $C \leftarrow C \bigcup \{t^\prime\}$\;
        }
    }
    \caption{Black-box mutational fuzzing}
\label{algo:bbfuzzing}
\end{algorithm}

The main algorithm works within a loop that stops at a predetermined condition
such as the end of a time budget, after the first crash has been found or after
a certain number of inputs have been tested, to name a few. More rudimentary
tools such as \textbf{zzuf}~\cite{hocevar2011zzuf} or
\textbf{Radamsa}~\cite{helin2015radamsa} allow for the tester to apply his or
her own stop criteria for the fuzzing campaign. The \texttt{SelectSeed} function
on line~\ref{algo:bbfuzzing:ss} of Algorithm~\ref{algo:bbfuzzing} selects one
input from the seeds corpus using the strategy of choice (\eg~stochastically, by
execution time, crash density). Next, the \texttt{MutateSeed} function applies a
mutation operator (\eg~bit-flips, insertion or deletion of words) to the
selected input to create a new one which is then fed to the \ac{SUT}. Different
mechanisms can then be deployed to identify whether the program crashed under
the given input, often program or \ac{OS} specific; if the \ac{SUT} exhibits a
fault, the test case is stored with useful information about the occurred fault
to later report about it to the tester.

Tools like zzuf or Radamsa only implement the \texttt{MutateSeed} function on
line~\ref{algo:bbfuzzing:ms}, leaving the remaining components' implementation
to the tester. For example zzuf applies random bit-flips to its input using a
set \emph{mutation ratio} (how much of the input to change) fully configurable
by the user within an interval or fixed. Radamsa performs instead a number of
more sophisticated mutation operators such as insert, repeat, drop and swap on
entities like bytes, ASCII and Unicode texts or arithmetic manipulations. More
complex black-box mutational fuzzers implement all components of
Algorithm~\ref{algo:bbfuzzing} and are able to exploit knowledge of the running
campaign to achieve better results.
The \textbf{\acf{BFF}}~\cite{householder2012probability} uses \emph{crash
density} as a metric to decide which pair of seed input and mutation ratio to
use for the next call to \texttt{MutateSeed} (\ac{BFF} uses indeed zzuff within
its mutation engine). Crash density of a seed is defined as the number of
crashes found by fuzzing that seed, divided by the number of total test cases
generated by the seed. Each execution of a mutated seed is modeled as a
Bernoulli trial where the outcome is whether or not the \ac{SUT} exhibited a
defect. The Binomial distribution that would result from successive trials is
approximated by a Poisson distribution as the number of trials is much higher
than the number of times a fault is found. The upper bound of the $95\%$
confidence interval of that distribution is then used to compute the probability
$p_i$ of selecting the seed file $t_i$. The same process is applied for a single
seed file and a fixed set of mutation ratio ranges so that for each seed file
there is a probability distribution over the set of ranges.
\citeauthor{woo2013scheduling} give the name \ac{FCS}~\cite{woo2013scheduling}
to describe the problem of selecting the next seed and mutation ratio pair to
fuzz (what they call a \emph{fuzzing configuration}) and recognize the
\ac{MAB}~\cite{berry1985bandit} nature of the problem. The authors take one step
further by modeling black-box mutational fuzzing as a weighted version of the
Coupon Collector's Problem and use those insights to inspect the \ac{FCS}
problem along three different axes that allows them to compose and evaluate a
total of 26 \ac{MAB} algorithms.

\section{Coverage-based Gray-box Fuzzing}
\label{sec:cgf}

A \ac{CGF} uses lightweight instrumentation and monitoring of the \ac{SUT} to
gain coverage information. This information is then exploited to provide a
solution to the \ac{FCS}
problem~\cite{afltech,lemieux2017fairfuzz,bohme2017directed,bohme2017coverage}.
The general approach is described in Algorithm~\ref{algo:cgf}.

\begin{algorithm}
    \DontPrintSemicolon%
    \SetKwFunction{SelectNext}{SelectNext}
    \SetKwFunction{AssignEnergy}{AssignEnergy}
    \SetKwFunction{MutateInput}{MutateInput}
    \SetKwFunction{IsInteresting}{IsInteresting}
    \SetKwInOut{Input}{Input}\SetKwInOut{Output}{Output}
    \Input{Set of seed inputs $S$}
    \Output{Set of crashing inputs $C$}
    \BlankLine%
    $C \leftarrow \emptyset$\;
    $Q \leftarrow S$\;
    \If{$Q = \emptyset$}{$Q \leftarrow \{\text{empty file}\}$}
    \Repeat{timeout reached or abort signal received}{%
        $t \leftarrow \SelectNext{Q}$\;\label{algo:cgf:sn}
        $p \leftarrow \AssignEnergy{t}$\;\label{algo:cgf:ae}
        \For{$i\in\left[0 \dots p\right]$}{%
            $t^\prime \leftarrow \MutateInput{t}$\;\label{algo:cgf:mi}
            \uIf{$t^\prime$ crashes}{%
                $C \leftarrow C \bigcup \left\{t^\prime\right\}$\;
            }
            \ElseIf{\IsInteresting{$t^\prime$}\label{algo:cgf:ii}}{%
                $Q \leftarrow Q \bigcup \left\{t^\prime\right\}$\;
            }
        }
    }
    \caption{Coverage-based Gray-box Fuzzing}
\label{algo:cgf}
\end{algorithm}

The first difference from black-box mutational fuzzing is that a \ac{CGF} does
not need a corpus of seed files to work properly (although it would be more
efficient). \citeauthor{afl}, the author of AFL, was able to generate valid JPEG
images starting from an empty file~\cite{afljpeg}. The functions
\texttt{SelectNext} and \texttt{MutateInput} (at lines~\ref{algo:cgf:sn}
and~\ref{algo:cgf:mi} respectively) are analogues of \texttt{SelectSeed} and
\texttt{MutateSeed} of Algorithm~\ref{algo:bbfuzzing}. The framework of
Algorithm~\ref{algo:cgf} incorporates explicitly the mechanics that \emph{smart}
black-box mutational fuzzers like \ac{BFF} implement. The function
\texttt{AssignEnergy} at line~\ref{algo:cgf:ae} decides how much effort should
be put in fuzzing the selected test-case (\eg~how many mutated inputs should be
created from it). Another difference from black-box mutational fuzzers is the
\texttt{IsInteresting} function at line~\ref{algo:cgf:ii}, responsible to
determine whether the mutated input is deemed \emph{interesting} and worth
fuzzing; this allows for \acp{CGF} to build a corpus of test-cases that could
even be reused with other tools or to fuzz another software that accepts the
same file format. For \acp{CGF}, interesting, loosely means that increases code
coverage and by keeping a queue of test-cases with ever-increasing coverage
helps fuzzers of this kind reaching deeper portions of the \ac{SUT} compared to
black-box mutational fuzzers.

\subsection{American Fuzzy Lop}
% - AFL-likes: AFL, AFLFast, FairFuzz
AFL~\cite{afl} is one of the most well known \acp{CGF}. Its focus is not on any
singular principle or insights but is rather a collection of hacks that have
been tested and proved effective in practice; the governing principles for its
development are speed, reliability and ease of use~\cite{afltech}. AFL gets its
coverage feedback from the \ac{SUT} by instrumenting its compiled binary. This
is done by injecting specific locations of the \ac{SUT} with a monitoring
snippet of code. This code can be injected by compiling the \ac{SUT} with the
\texttt{afl-gcc} utility or, when the source code is not available, AFL uses
QEMU~\cite{bellard2005qemu} to dynamically (during interpretation, at runtime)
instrument the binary. AFL-dyninst~\cite{afldyn} is an extension to AFL that
injects the instrumentation snippet directly into the binary. What AFL injects
into the \ac{SUT} is essentially equivalent to that presented in
Listing~\ref{lst:aflbininst}.

\begin{lstlisting}[caption={AFL's instrumentation},label=lst:aflbininst,float]
    cur_location = <COMPILE_TIME_RANDOM>;
    shared_mem[cur_location ^ prev_location]++;
    prev_location = cur_location >> 1;
\end{lstlisting}

The instrumentation snippet is injected at every branch within the instrumented
code. The \texttt{cur\_location} variable is generated randomly at compile time
and identifies the current \emph{basic block} (a straight code sequence without
branches besides at its entry and exit points). The \texttt{shared\_mem} array
is a 64KB shared memory region provided by the fuzzer; each byte of the shared
memory can be thought of as a hit to a transition from one branch to another.
The shift operation at the last line preserves directionality of the tuples
(\eg~$A \oplus B$ would be indistinguishable from $B \oplus A$) and to keep the
identity of loops within the same basic block (\eg~$A \oplus A$ would be equal
to $B \oplus B$).

With regards to Algorithm~\ref{algo:cgf}, AFL implements the \texttt{SelectNext}
function by classifying elements of the queue as \emph{favorites}. A test-case
is deemed favorite if it exhibits faster speed of execution and small size for
the branch tuples that it covers compared to the other test-cases in the queue.
AFL selects favorite items more often, implementing a strategy that favors,
using the \ac{MAB} terminology, exploitation against exploration
\cite{bohme2017coverage}. AFL initially applies a set of deterministic mutation
operators to selected inputs and later uses what it calls an \emph{havoc stage}
where more complex and stacked mutations are stochastically applied; the
function \texttt{AssignEnergy} determines how many mutations should be applied
to the selected test-case during the havoc stage. AFL's implementation of
\texttt{AssignEnergy} uses a mix of execution speed, coverage information and
age of the selected test-case to determine how many inputs should be generated
by mutating it. AFL implements a good number of mutation operators such as
bit-flips, insertion and deletion of bytes, arithmetic operators, to name a few
\cite{aflmut}. The function \texttt{IsInteresting} as implemented by AFL,
returns true if the input $t^\prime$ exercises a new basic block transition or
if the number of times a transition is hit goes from one range of values to the
next; AFL employs a ``bucketing'' scheme where the range of values required to
be in the next bucket roughly doubles (the exact values are $1$, $2$, $3$,
$4-7$, $8-15$, $16-31$, $32-127$ and $128+$)~\cite{afltech}.

\paragraph{AFLFast}
By observing that an high proportion of test-cases generated by AFL exercise a
small number of \emph{high frequency paths}, \citeauthor{bohme2017coverage} draw
the intuition that steering fuzzing more toward \emph{low frequency paths} could
lead to an improvement in performance (exploring more paths with the same amount
of fuzz). AFLFast~\cite{bohme2017coverage} is an extension to AFL that
implements two search strategies within the \texttt{SelectNext} function and a
number of power schedules within \texttt{AssignEnergy}. To derive more
meaningful insights, the authors model Coverage-based Gray-box Fuzzing as a
Markov Chain and use their results to dissect problems of \acp{CGF} and derive
meaningful power schedules (an exploitation-based constant schedule, which is
the same used by AFL, an exploration-based constant schedule and cut-off
exponential, exponential, linear and quadratic schedules). Their evaluation of
AFLFast over common UNIX utilities shows that the exponential schedule works
best and provide promising results when compared to AFL\@. When both AFL and
AFLFast are able to discover (within the given time budget) the same
vulnerability, AFLFast does it much quicker.
% maybe expand on the CGFuzzing as a Markov Chain?

\paragraph{FairFuzz}
\citeauthor{lemieux2017fairfuzz} devise another extension to AFL that targets
\emph{rare branches} with the aim to drive the fuzzing process deeper into the
\ac{SUT}~\cite{lemieux2017fairfuzz}. FairFuzz changes how the
\texttt{SelectNext} and \texttt{MutateInput} functions of
Algorithm~\ref{algo:cgf} work. When selecting inputs from the queue, FairFuzz
prioritizes test-cases that exercise a rare branch; successively new fuzz is
produced by applying byte-level mutation operators to the selected input trying
to exercise the same rare branches while still exploring new parts of the
\ac{SUT}. The intuition behind this is that for most file formats, there are
sequences of bytes acting as headers to prove the validity of the file format
and other bytes that trigger execution of various parts of the program that
processes the file. Among all the fuzz generated by AFL, only a few will contain
the right sequence of bytes in the right place; FairFuzz may classify branches
triggered by this header as rare and apply mutation operators that preserve the
header in the generated fuzz. More specifically a branch is rare if it is hit by
a number of inputs (amount of fuzz) smaller than the \emph{rarity cutoff}. This
threshold is computed after each call to \texttt{SelectNext} as $2^i$ where
$2^{i-1} < \min(B_{hit}) \le 2^i$ and $B_{hit}$ is the set of hit counts for all
branches discovered so far. At the core of the mutation operators instead, the
authors describe the concept of \emph{branch mask}, an artifact used to
determine at which positions in the input bytes can be overridden, deleted or
inserted. The branch mask, computed during the deterministic stage of mutation,
is then used to steer mutation in the havoc stage so that rare branches are
still covered by the mutated input.

\subsection{Honggfuzz}
Honggfuzz~\cite{honggfuzz} is another general-purpose \ac{CGF} which simplifies
the semantics of Algorithm~\ref{algo:cgf} but offers state-of-the-art
implementation that grants huge throughput (generates lots of fuzz, especially
in persistent mode). The most peculiar feature of Honggfuzz is allowing the user
to gather feedback from the \ac{SUT} via software or hardware sources; it
supports CPU branch and instruction counting by means of Intel \ac{BTS} or Intel
\ac{PT} on supported CPUs. Honggfuzz selects uniformly at random one input to
mutate from the queue and assigns constant energy of $1$ to it. The
\texttt{MutateInput} function picks uniformly at random one of the implemented
mangling functions and applies it to the selected input. Honggfuzz considers a
new input interesting if the used coverage metric increases; if using Intel
\ac{BTS} for example, Honggfuzz maintains a bitmap containing branch coverage
information (similar to the one used by AFL) and any fuzz exercising a
previously unseen branch is considered interesting and added to the queue.

\subsection{VUzzer}

\section{Cooperative Fuzzing}
\label{sec:coop}
% TODO: using different fuzzers / methods together
% - driller & symfuzz
% - OSS-fuzz & clusterfuzz
% - Chemotactic

% TODO: brief of how we tackle the problem of using different fuzzers

