%*******************************************************
% Abstract
%*******************************************************
%\renewcommand{\abstractname}{Abstract}
\pdfbookmark[1]{Abstract}{Abstract}
\begingroup%
\let\clearpage\relax%
\let\cleardoublepage\relax%
\let\cleardoublepage\relax%

\chapter*{Abstract}

\emph{Fuzzing} is a popular technique for testing software for reliability and
security. As different fuzzers are specialized to different kinds of software
and make different assumptions about it, the practitioner is often tasked to
select the appropriate fuzzer for the \sut. Otherwise, if enough resources are
available, they choose a set of fuzzer to run --- \emph{independently} --- in
parallel or sequentially.

In this thesis we present a \ac{CFF} that allows a set of fuzzers, running in
parallel, to communicate and exchange information. We describe a distributed
implementation of the framework that uses hardware-generated coverage feedback
to control the flow of information among the fuzzer instances. Moreover, the
system is designed to integrate with a generic fuzzer that implements an
\acs{API} which is already implemented by most fuzzers.

We evaluate the \ac{CFF} using four popular fuzzers on four UNIX utilities. The
results show promising improvements both in terms of code coverage and unique
crashes found.

\endgroup%

\vfill%

